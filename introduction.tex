\section{Introduction}
\label{sec:introduction}
\subsection{Introduction}
Introduction here 

From meeting:
Assumption: climate is stationary.
Do not focus on wrong or wright with the 9g vs 1g model. All the methods are alternatives. Discuss how you did it with the models and how to interepret the results they give. Use the word "concistency" rather than right or wrong. One version of the model may be more consistent with the station based idf values than others.
Try to aim for: wuantifieng change in percent from 1g to the various 9g methods. Discuss the change of results based on method.
Also include 9g indivdual rom a couple of stations to highlight the individual station uncertanty (topography play an inportant role in small scale conventve evetns). Discuss and cuantify uncertanty. 
Inclue sruface scheme (one paragraph) in the AROME descripton.
Discuss the method with ANita, and then Jana for input. 

\subsection{Research aim?}
Firstly the main goal for this thesis was to investigate how future IDF values differ from values obtained by historic station data. The idea was to verify the general opinion on increased precipitation extremes through the IDF curved derived by modelled data. One step towards this goal is to verify whether the modelled precipitation can at all be used for this purpose. To check this, I must first evaluate the IDF-values against some stations. Along the way I must define what return periods are realistically represented by the data. If they match I can proceed to do an analysis of other locations. If not I should investigate why, or to what extent these data can be used in terms of IDF values in Norway.

\\
For an introduction: get inspiration in text from orskar and lind about model-setup of HARMONIE: \cite{lind_arome} 