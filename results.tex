\section{Results}
\label{sec:results}

\subsection{Preliminary findings}

To find the best model-fit to the station data for the historical period a 9-point grid around the grid closest to the station is used. The maximum precipitation value for each time step is chosen out of the 9 grid points. This is done to minimize the risk of missing an modelled event next to the station grid-point. An event modelled at a certain grid-point might as well happen at the grid-point next to it. To evaluate if this grid selection is better than choosing the grid-point closest to the stations the absolute average difference between the station values and the model values for all durations is calculated for both the 1 grid-procedure and the 9-grid-procedure. If the difference is 0 it means the modelled values are equal to the station-values.  

\begin{table}
\begin{tabular}{ c c c c }

Duration & 1grid & 9grid & diff 1g-9g \\
2 & 2.93 & 8.22 & -5.30 \\
5 & 5.62 & 8.54 & -2.92 \\
10 & 7.86 & 8.93 & -1.07 \\
20 & 10.23 & 9.66 & 0.57 \\
25 & 11.04 & 9.96 & 1.08 \\
50 & 13.71 & 11.19 & 2.52 \\
100 & 16.68 & 12.73 & 3.95 \\
200 & 20.02 & 14.58 & 5.44
\end{tabular}
    \label{grid}
\end{tabular}
\caption{Grid selection. Positive difference: 9grid is better. Negative: 1grid is better}
\end{table}

As displayed is Table \ref{grid} the 9grid-selection outperforms the regular 1grid-selection on the larger return periods, while it is the opposite for smaller return periods.
\textbf{Whys is this the case?}

\subsection{Surrounding grid points Blindern station}

It turned out that the "1grid" approach underestimated the returnvalues, while the "9grid" approach overestimated them. I have calculated the mean difference between the station Blindern and the individual gridpoints surrounding the station to check if any of the gridpoints have particular large impact on the maximum value chosen per time step in the "9g" analysis. The gridpoint to the south and south-west where further away from the station than the other directions by a huge margin. For some return periods the difference where more than twice as large as the smallest differnece for that return period. These directions may affect the overall "9grid" method too much.   

In the "retlev" values in "IDF ECE Blindern 9grid.csv" is smaller for almost all directions and return periods. This is a mismatch with the resulting figures from the 9grid selection method. For all return periods in these figure the Blindrn return values are alot higher for the modeled data over the station data. So why are all return values from the grid points around blindern lower than the station? something is incorrect.

The individual 9g grid points have very lov mean return values for all return periods compared to the first 9g method. Should be the same. 

\subsection{9grid mean vs max method}
Previously the selection method was based on choosing the maximum value per time-step out of the nine gridpoints, forming a one-dimensional timeseries. Then the annual maxima was extracted for each duration. This approach was designed to maximize the chance of capturing an extreme event in close proximity to the station. As discussed above (?) this method appears to "optimize" the modeled precipitation and hence overestimate the return values. Retunvalues for all stations and almost all durations are within the 95 percentile of the station-based retunvalues for the larger returnperiods. This is not surprising given the enormous confidence interval in the largest retunperiods at up to around 200 mmm for the largest durations for some stations. For the smaller returnperiods like 2 and 5 years the method is outside the 95 percentile of the station-based returnvalues. Here the confidence interval is of coarse noticeably smaller. \textbf{make sure its even relevant to comapre this against the confidence interval to the stat based values. Justify it somewhere}.  

make a comment on how the 9grid first method follows the large (100 and 200) returnperiods station idf curves very well for the stations with large confidence intervals. Especially Bygdøy and Besserud and partially Haugenstua (for large durations on Haugenstua).


\textbf{Here (or later) you can discuss what the station is actually representing. It might be that the $XXX$ method is very representative for the area in general. It would be interesting to check how the station-based IDF curves compares to observations. If these curves are underestimating the observations $XXX$ might be very well suited to represent the overall conditions fr extreme precipitation in the area. Also, describing the methods for extraction should be in the methods section, not here in results}

The resulting $MEAN$ method provides very similar returnvalues to the $1GRID$ method for all stations and all durations. For short returnperiods like 2 or 5 years the returnlevels are close to identical to the $1GRID$ method for all durations and stations, while for the larger returnperiods like 100 or 200 years the $MEAN$ method yields slightly smaller returnvalues for durations up to around 3 hours. This difference appears a smoother inclination of returnvalues compares to the very flat evolution of the $1GRID$ returnvalues for durations between 90 minutes and 3 hours. Out of all the methods the $1GRID$ and the $MEAN$ methods are clearly producing the smallest returnlevels for all stations and durations, and they are consequently smaller compared to the station-based values. Both the $MEAN$ and the $1GRID$ have almost identical returnvalues for all durations on short returnperiods for stations 18701 Blindern, 18320 Hausmansgate and 18270 Vestli. 

\begin{figure}[hbt!]
    \centering
    \includegraphics[scale=0.4]{figures/200_retper_ECE_1985_compare.png}
    \caption{Example figure. This is what I am talkin about
    \cite{lind_arome}}
    \label{fig:arome_domain}
\end{figure}

On average for all stations the $MAX$ method overestimate the IDF-values slightly for 2 year returnperiod, and slightly underestimate for all returnperiods larger than 5 years. In table \ref{dist_eval} the realtionship between the respective model and the station-based returnvalues are listed in station- and duration-averaged percentage. For station Haugenstua, Ljabruvegen, Hovin and Besserud it is close to identical to the $STAT$ method for all durations on the 2 year returnperiod. For the larger returnperiods it is the overall method closest to the $STAT$ method, esepecially for the stations with narrow confidence intervals. Providing higher estimates of the returnvalues than the $MEAN$ and the $1GRID$ method but smaller estimates compared to the $9GRID$ the $MAX$ method serves as a middle ground which according to table \ref{dist_eval} is more consistent with the station-based estimates on average for all the Oslo stations. Only for a 200 year returnperiod the station average for the $9GRID$ method is more consistent.        

\begin{table}[hbt!]
\centering
\begin{tabular}{ c c c c c}

Duration & 1grid & 9grid & mean9 & max9 \\
2 & 88.78 & 134.92 & 89.30 & 107.89\\
5 & 82.27 & 125.45 & 79.57 & 100.00\\
10 & 79.29 & 121.10 & 75.09 & 96.37\\
20 & 77.02 & 117.78 & 71.65 & 93.60\\
25 & 76.39 & 116.86 & 70.70 & 92.84\\
50 & 74.66 & 114.33 & 68.07 & 90.73\\
100 & 73.20 & 112.20 & 65.83 & 88.95\\
200 & 71.93 & 110.36 & 63.89 & 87.42
\end{tabular}
\caption{Station- and duration-average returnvalues for each returnperiod in percentage of the station-based station- and duration-average returnvalues. Each column is one method. Values >100 means averge overestimation copared to to the station-based values and values <100 mean underestimation.}
\label{dist_eval}
\end{table}


Even though we (I and Malte) agreed on not using the distance between the curves as any measure, it must be said that the "ax9" method now gives far better overall results compared to the stat.

\subsection{Annual Maxima}

Investigating how the annual maxima time-series of the stations and the model behave might provide information on the nature of the IDF values and their uncertainties. Figure \ref{fig:Am_stations} shows AM for all stations for 15 minutes and 24 hours duration for station 18701 Blindern. The black line are the mean AM precipitation value. Even though Figure \ref{fig:AM_stations} only show two of the durations, analysis show large variations in AM between stations across all durations for most stations. The spread for each yearly value seems to be quite large. A few years like 2003 and 2004 have available station AM values close to each other for all durations. This also apply to other years, but in most cases the number of data-points for those years are only 2-3. From 1968 to 1996 there are mostly three stations available each year. In this period some years have one or two stations available. Thus, four to five stations governs the AM pattern for all durations more than half of the total time-series length from 1968 to 2018. From around year 2000 the number of available stations each year increases. This also increases the spread in this years. Some stations have largest or highest AM value a specific year for some or all durations, but this is more not than often the case.      

\begin{figure}[hbt!]
    \centering
    \includegraphics[scale=0.8]{figures/AM_stations.png}
    \caption{A caption.}
    \label{fig:AM_stations}
\end{figure}
\\
\\
Now we investigate how the modelled AM precipitation compares to the station AM. In Figure \ref{fig:AM_stat_mod} the station mean AM from Figure \ref{fig:AM_stations} is plotted with station average AM from the different AM methods for durations 15, 60, 360 and 1440 minutes from 1985 to 2005. The shaded gray area is one standard deviation (STD) of the station mean AM for the respective duration. The most noticeable feature is that the four methods used with the modelled precipitation overlap quite well with the station-curve. An exception that with longer duration the 9GRID method appears to fall increasingly far outside the 1 STD of the station-curve. For all durations 1GRID and 9MEAN AM are close to identical throughout the 20 years, while MAX falls between 9GRID and MEAN/1GRID. In periods like 1988-1992 and 1995-2000 most durations up to 360 minutes the modelled AM and the station-based AM agree very well, while the durations 360 minutes and larger have a good agreement between the model and the measurements in the period 1989-1995. For almost all years and duration one or more of the modelled methods are within the standard deviation of the measurements. However a few exceptions exists, like the year 2001 where all modeled AM methods are outside the measurement STD for durations up to 360 minutes. Another noticeable feature is that the 9GRID method appears to increasingly overestimate the AM values with increasing duration. This might suggest hat the method is not very well suited for IDF calculations, especially for the larger durations.   

\begin{figure}[hbt!]
    \centering
    \includegraphics[scale=0.8]{figures/AM_stat_mod.png}
    \caption{A caption.}
    \label{fig:AM_stat_mod}
\end{figure}
\\
\\
Annual maxima differs substantially between the methods for the individual stations and durations. The variations in AM could partially account for the spread in the resulting IDF-curves. In Figure \ref{fig:AM_dur} a heatmap of the duration average variance of annual maximum precipitation for each station and each method is plotted. The number within the parenthesis on the y-axis refers to the time-series length of the station-data. For the other methods the time-series length is 20 years for all stations. The most noticeable feature is the large range of variance on the station-based data compared to the other method. Large variance indicates large spread, and hence the year to year difference in annual maxima is larger for the station data compared to the modelled data. Station 17980 Ljabruvegen has the smallest STD at XXX while station 18815 Bygdøy has the largest STD at for the station-based annual maxima. It is also noticeable how the STD for stations like 18701 Blindern and 18020 Lambergseter with long time series is on the high end of the scale amongst these stations. At the same time short time-series stations like 18980 Lilleaker, 18920 Besserud and 18640 Vestre Vika also have high relative STD.  

\begin{figure}[hbt!]
    \centering
    \includegraphics[scale=0.6]{figures/AM_dur_avg_std.png}
    \caption{Duration average variance of annual maximum precipitation data.}
    \label{fig:AM_dur}
\end{figure}

The MEAN method has a very consistent, low STD at around 4 mm across all stations. Also MAX and G1 has low STDs, but the differences between the stations are greater. The largest STD for the MEAN and G9 method are found at station 18701 and the largest for the G1 and STAT are found at station 18815. \textbf{DISCUSSION: 18701 has a long time series while 18815 has a short. Thus its not obvious that the time series length is closely linked to the STD of the AM.}

Figure \ref{fig:AM_stat} shows a heatmap of station average STD of annual maximum precipitation for the different AM methods. All durations are listed on the y-axis while the methods are listed on the x-axis. The first noticeable feature of the plot is the increasing STD with increasing duration across all methods. Only with exception at duration 180 minutes for the G1 method and at 360 minutes for the MAX method the STD are increasing with increasing duration. MEAN has the overall smallest STD with 7.66 mm as maximum value. MAX and G1 has very similar STD across the duration. G9 is the method with STD most similar to STAT.   

\begin{figure}[hbt!]
    \centering
    \includegraphics[scale=0.4]{figures/AM_station_avg_std.png}
    \caption{Station average STD of annual maximum precipitation data.}
    \label{fig:AM_stat}
\end{figure}

\subsubsection{Station IDF}

A version of the IDF-curves are presented in Figure \ref{fig:IDF_stat_retper}. Here station-based precipitation magnitude are plotted for all durations at all stations for 2, 20 and 200 year recurrence interval. Despite a relatively short distance between the stations of a few kilometers there are relatively large differences in expected precipitation between them. Station Besserud and station Blindern is only 3km apart, however the 5 year return period return-level at 24h is already more than 10 mm. For the 200 year return period this difference is around 50 mm. In general the difference between high and low return-levels across the stations are larger for larger return periods and smaller for smaller return periods.

\begin{figure}[hbt!]
    \centering
    \includegraphics[scale=0.3]{figures/IDF_stat_retper.png}
    \caption{Station average STD of annual maximum precipitation data.}
    \label{fig:IDF_stat_retper}
\end{figure}

Stations like Ljabruvegen and Blindern have a small spread between the 2 year and the 200 year return-level for all durations, while stations like Haugenstua and Besserud has a large spread between the return-levels. The varying station length may contribute to this difference. Blindern has 48 years of data and a small spread compared to Besserud with 13 years of data, thus it appear that the time-series length has a great impact on the spread of the return-levels for each station. On the other hand Ljabruvegen has 17 years of data and a small spread compared to Haugenstua with 15 years of data, so the time-series length is not the only thing affecting the return-levels.   

\textbf{Include figure with real IDF for stat all stations? Like the one above but with intensity} 

\subsubsection{Introducing modelled precipitation}


\subsection{2080-2100}
